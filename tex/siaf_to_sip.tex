\documentclass[10pt]{article}
\usepackage[legalpaper, portrait, margin=1in]{geometry}
\usepackage[scaled]{helvet}
\usepackage[T1]{fontenc}
\renewcommand{\familydefault}{\sfdefault}
\newcommand{\sipA}[2]{\ensuremath{\mathtt{A}_{#1 #2}}}
\newcommand{\sipB}[2]{\ensuremath{\mathtt{B}_{#1 #2}}}
\newcommand{\sipAP}[2]{\ensuremath{\mathtt{AP}_{#1 #2}}}
\newcommand{\sipBP}[2]{\ensuremath{\mathtt{BP}_{#1 #2}}}
\newcommand{\siafXP}[2]{\ensuremath{X^{T}_{#1 #2}}}
\newcommand{\siafYP}[2]{\ensuremath{Y^{T}_{#1 #2}}}

%\newcommand{\NCu}{\ensuremath{NC_{11}}}
%\newcommand{\NCv}{\ensuremath{NC_{22}}}
\newcommand{\NCu}{\ensuremath{\mathtt{PC1\_1}}}
\newcommand{\NCv}{\ensuremath{\mathtt{PC2\_2}}}
\newcommand{\CDu}{\ensuremath{\mathtt{CD1\_1}}}
\newcommand{\CDv}{\ensuremath{\mathtt{CD2\_2}}}
\title{SIAF to SIP Conversion}
\author{Brant Robertson}
\begin{document}
\maketitle
\section{Overview}
We want to determine the translation between the SIAF and SIP 
polynomial distortion models. Below, we define the expansion models
and then identify like terms. We will write down the relation between
the SIAF and SIP expansion coefficients succinctly at the end.

\section{SIP Expansion Definition}
The Simple Imaging Polynomial (SIP) expansion is
defined in terms of pixel coordinates $u,v$ relative to 
an origin at {\tt CRPIX1}, {\tt CRPIX2} located on sky
at {\tt CRVAL1}, {\tt CRVAL2}.  The on-sky location $x,y$
of pixel $u,v$ is given by
\begin{equation}
\label{eqn:sip_expansion}
\left(\begin{array}{c}x\\y\end{array}\right) = \left(\begin{array}{cc} \mathtt{CD1\_1}&\mathtt{CD1\_2}\\\mathtt{CD2\_1}&\mathtt{CD2\_2}\end{array}\right)\left(\begin{array}{c} u + f(u,v)\\v+g(u,v)\end{array}\right).
\end{equation}
\noindent
Note that the {\tt CD} matrix contains all the units (of pixel scale).

The functions $f(u,v)$ and $g(u,v)$ define the polynomial expansions that provide the distortion. These
functions are given by
\begin{eqnarray}
\label{eqn:sip_f}
f(u,v) = \sum_{p,q} \sipA{p}{q} \,u^p \,v^q, & p+q \leq \mathtt{A\_ORDER},
\end{eqnarray}
\begin{eqnarray}
\label{eqn:sip_b}
g(u,v) = \sum_{p,q} \sipB{p}{q} \,u^p \,v^q, & p+q \leq \mathtt{B\_ORDER},
\end{eqnarray}
\noindent
where $\sipA{p}{q}$ and $\sipB{p}{q}$ are the coefficients that encode the degree of 
distortion, and $\mathtt{A\_ORDER}$
and $\mathtt{B\_ORDER}$ limit the number of terms in the distortion model. Note that the sum $p+q$ is less
than or equal to $\mathtt{A\_ORDER}$ (for $f(u,v)$) or $\mathtt{B\_ORDER}$ (for $g(u,v)$). In a record of the
SIP coefficients, any unspecified coefficient is assumed to be zero.

The original SIP definition only included quadratic and higher terms, such that third-order model would be written
\begin{eqnarray}
f_{old}(u,v) = \sipA{2}{0}u^2 + \sipA{0}{2}v^2 + \sipA{1}{1}uv + \sipA{2}{1}u^2 v + \sipA{1}{2}uv^2 + \sipA{3}{0}u^3 + \sipA{0}{3}v^3,\\
g_{old}(u,v) = \sipB{2}{0}u^2 + \sipB{0}{2}v^2 + \sipB{1}{1}uv + \sipB{2}{1}u^2 v + \sipB{1}{2}uv^2 + \sipB{3}{0}u^3 + \sipB{0}{3}v^3.
\end{eqnarray}
\noindent
However, we will need the linear cross terms
\begin{eqnarray}
\label{eqn:fg}
f(u,v) = \sipA{0}{1}v + \sipA{2}{0}u^2 + \sipA{0}{2}v^2 + \sipA{1}{1}uv + \sipA{2}{1}u^2 v + \sipA{1}{2}uv^2 + \sipA{3}{0}u^3 + \sipA{0}{3}v^3,\\
g(u,v) = \sipB{1}{0}u +\sipB{2}{0}u^2 + \sipB{0}{2}v^2 + \sipB{1}{1}uv + \sipB{2}{1}u^2 v + \sipB{1}{2}uv^2 + \sipB{3}{0}u^3 + \sipB{0}{3}v^3.
\end{eqnarray}
\noindent
Note that we expect $\sipA{1}{0}=0$ and $\sipB{0}{1}=0$ because they are degenerate with redefining the
{\tt CD} matrix elements.

To finish our SIP description, we should multiply through by the {\tt CD} matrix. It will be convenient
for us to use a coordinate system where the off diagonal terms are zero (aligned with the detector) and
decompose the {\tt CD} matrix as
\begin{equation}
\left(\begin{array}{cc} \mathtt{CD1\_1}&\mathtt{CD1\_2}\\\mathtt{CD2\_1}&\mathtt{CD2\_2}\end{array}\right) = 
\left(\begin{array}{cc} \mathtt{CDELT1} \cdot  \NCu & 0\\0 &\mathtt{CDELT2} \cdot \NCv \end{array}\right).
\end{equation}
\noindent
Note that $\NCu$ and $\NCv$ account for a possible parity change between the detector and sky coordinates (they are equal to $\pm1$ depending on the detector),
and this decomposition only helps us because of convenient definitions made by STScI in their SIAF expansion (see below).

We can then write the SIP distortion as
\begin{equation}
\label{eqn:sip_x}
x = \mathtt{CDELT1} \cdot  \NCu \left[u + f(u,v)\right],
\end{equation}
\noindent
\begin{equation}
\label{eqn:sip_y}
y = \mathtt{CDELT2} \cdot  \NCv \left[v + g(u,v)\right],
\end{equation}
\noindent
where $f(u,v)$ and $g(u,v)$ are given to, e.g., third order by Equations \ref{eqn:fg} and 7.

\section{SIAF Expansion Definition}

The SIAF expansion is given by STScI and defined in JWST Technical Report {\tt JWST-STScI-001550} (hereafter JTR1550). There
are a few steps involved, and a few choice cancellations make things a bit easier for NIRCam than they
could be!

First, there is a translation between the {\tt Detector} coordinates ($u$ and $v$ from above) to the
{\tt Science} coordinates. There is a further translation from {\tt Science} to {\tt Ideal} coordinates
that involves the SIAF polynomial expansion. It is important to note that the SIAF expansion coefficients
are \emph{dimensionful}.

\subsection{Detector to Science}

The detector to science coordinate conversion is given in \S 4.1 of JTR1550 as
\begin{eqnarray}
XSci - XSciRef &=& DetSciParity\left[\left(XDet - XDetRef\right)\cos\left(DetSciYAngle\right)\right.\\
               &&+ \left.\left(YDet - YDetRef\right)\sin\left(DetSciYAngle\right)\right],\nonumber
\end{eqnarray}
\begin{eqnarray}
YSci - YSciRef &=& -\left(XDet - XDetRef\right)\sin\left(DetSciYAngle\right)\\ 
               &&+ \left(YDet - YDetRef\right)\cos\left(DetSciYAngle\right),\nonumber
\end{eqnarray}
\noindent
where the absurd notation is in the original.  Note that for all NIRCam SCAs, $XDetRef=XSciRef=1024.5$
and all $YDetRef=YSciRef=1024.5$, so if we pick $\mathtt{CRPIX1}=1024.5$, $\mathtt{CRPIX2}=1024.5$,
then $u\equiv\left(XDet - XDetRef\right)$ and $v\equiv\left(YDet - YDetRef\right)$. Also, for
all NIRCam SCAs, $DetSciParity=-1$. Lastly, for NIRCam SCAs we have either $DetSciYAngle=0$ or
$DetSciYAngle=\pi$. We then can write
\begin{eqnarray}
XSci - XSciRef &=& DetSciParity\cos\left(DetSciYAngle\right) u
\end{eqnarray}
\begin{eqnarray}
YSci - YSciRef &=& \cos\left(DetSciYAngle\right) v
\end{eqnarray}
\noindent
Let us define $\NCu \equiv DetSciParity\cos\left(DetSciYAngle\right)$ and 
$\NCv \equiv \cos\left(DetSciYAngle\right)$, so we are left with
\begin{eqnarray}
XSci - XSciRef &=& \NCu u
\end{eqnarray}
\noindent
and
\begin{eqnarray}
YSci - YSciRef &=& \NCv v.
\end{eqnarray}

\subsection{Science to Ideal}
The SIAF expansion is written in terms of a conversion between the {\tt Science}
and {\tt Ideal} coordinates as
\begin{equation}
\label{eqn:siaf_xidl}
XIdl = \sum_{i=1}^{degree} \sum_{j=0}^{i} Sci2IdlCoefX_{i,j}\left(XSci-XSciRef\right)^{i-j}\left(YSci-YSciRef\right)^j
\end{equation}
\begin{equation}
\label{eqn:siaf_yidl}
YIdl = \sum_{i=1}^{degree} \sum_{j=0}^{i} Sci2IdlCoefY_{i,j}\left(XSci-XSciRef\right)^{i-j}\left(YSci-YSciRef\right)^j
\end{equation}
\noindent
where $degree=Sci2IdlDegree$ (a parameter given in the SIAF file for each detector)
and again the notation is originally absurd. To give ourselves a break, we will 
define $X_{ij} \equiv Sci2IdlCoefX_{i,j}$ and $Y_{ij} \equiv Sci2IdlCoefY_{i,j}$ and
identify $XIdl\equiv x$ and $YIdl\equiv y$.

Since the SIAF expansion is dimensionful, to match terms with the SIP expansion
it will be helpful for us to write out fully the terms and then infer a recursion
relationship between the coefficients in each expansion. Given Equations
\ref{eqn:siaf_xidl} and \ref{eqn:siaf_yidl}, and the equivalences defined in the
previous paragraph, we can write the expansions to, e.g., $degree=3$ as
\begin{eqnarray}
\label{eqn:siaf_x}
x &=& X_{10} \NCu u + X_{11} \NCv v + X_{20} \NCu^2 u^2 \\
  &&+ X_{21}\NCu \NCv uv + X_{22} \NCv^2 v^2 + X_{30} \NCu^3 u^3 \nonumber \\
  && + X_{31}\NCu^2 \NCv u^2 v + X_{32} \NCu \NCv^2 u v^2  + X_{33} \NCv^3 v^3\nonumber,
\end{eqnarray}
\noindent
and
\begin{eqnarray}
\label{eqn:siaf_y}
y &=& Y_{10} \NCu u + Y_{11} \NCv v + Y_{20} \NCu^2 u^2 \\
  &&+ Y_{21}\NCu \NCv uv + Y_{22} \NCv^2 v^2 + Y_{30} \NCu^3 u^3 \nonumber \\
  && + Y_{31}\NCu^2 \NCv u^2 v + Y_{32} \NCu \NCv^2 u v^2  + Y_{33} \NCv^3 v^3\nonumber.
\end{eqnarray}

\section{Reconciling the SIAF and SIP Expansions}

We can now match terms and infer the relationship between the SIP and SIAF coefficients by
setting the right hand sides of Equations \ref{eqn:sip_x} and \ref{eqn:sip_y} equal to the
right hand sides of Equations \ref{eqn:siaf_x} and \ref{eqn:siaf_y} and grouping terms by
their polynomial order in the expansion. Let us start with the $x$ expansion,
\begin{eqnarray}
\label{eqn:long_x}
\mathtt{CDELT1} \cdot  \NCu u  &&\\
+ \mathtt{CDELT1} \cdot  \NCu \left[  \sipA{0}{1}v + \sipA{2}{0}u^2 + \sipA{0}{2}v^2 \right.&& \nonumber\\
\left. + \sipA{1}{1}uv + \sipA{2}{1}u^2 v + \sipA{1}{2}uv^2 + \sipA{3}{0}u^3 + \sipA{0}{3}v^3\right] \nonumber&&\\ 
&=& X_{10} \NCu u + X_{11} \NCv v + X_{20} \NCu^2 u^2 \nonumber\\
  &&+ X_{21}\NCu \NCv uv + X_{22} \NCv^2 v^2 + X_{30} \NCu^3 u^3 \nonumber \\
  && + X_{31}\NCu^2 \NCv u^2 v + X_{32} \NCu \NCv^2 u v^2  + X_{33} \NCv^3 v^3\nonumber.
\end{eqnarray}
\noindent
We want to express the SIP coefficients in terms of the
SIAF coefficients.  We will define for the SIAF
coefficients that $\tilde{X}_{ij} = X_{ij}/\mathtt{CDELT1}$.
We can identify the correspondence between coefficients as
\begin{eqnarray}
\label{eqn:matched_x}
X_{10} &\equiv& \mathtt{CDELT1} \\
\sipA{0}{1} &\equiv& \tilde{X}_{11} \NCu^{-1} \NCv \\
\sipA{2}{0} &\equiv& \tilde{X}_{20} \NCu \\
\sipA{0}{2} &\equiv& \tilde{X}_{22} \NCu^{-1} \NCv^{2} \\
\sipA{1}{1} &\equiv& \tilde{X}_{21} \NCv \\
\sipA{2}{1} &\equiv& \tilde{X}_{31} \NCu \NCv \\
\sipA{1}{2} &\equiv& \tilde{X}_{32} \NCv^{2} \\
\sipA{3}{0} &\equiv& \tilde{X}_{30} \NCu^{2} \\
\sipA{0}{3} &\equiv& \tilde{X}_{33} \NCu^{-1} \NCv^{3}.
\end{eqnarray}
\noindent
Given that we know $\NCu^2 = 1$, we can multiply both sides by unity and rewrite these
equations conveniently as
\begin{equation}
\label{eqn:siaf_to_sip_x}
\sipA{i}{j} = \tilde{X}_{(i+j) j} \NCu^{(i+1)} \NCv^j.
\end{equation}
\noindent
Note that there is no $\sipA{1}{0}$ term in the expansion
as the linear order term in $u$ in the SIP expansion is
already handled by multiplying the $\mathtt{CD}$ matrix
by $(u +f, v + g)$, so $\sipA{1}{0} = 0$. Using that
$\NCv^2 = 1$, 
we can write similar equations for the $y$ coordinate as
\begin{equation}
\label{eqn:siaf_to_sip_y}
\sipB{i}{j} = \tilde{Y}_{(i+j) j} \NCu^{i}\NCv^{(j+1)},
\end{equation}
\noindent
where $\tilde{Y}_{ij} \equiv Y_{ij}/\mathtt{CDELT2}$.
We note that $Y_{11} = \mathtt{CDELT2}$, and
$B_{01} = 0$ because the linear order $v$ expansion
term is handled by $\mathtt{CD}$ matrix
by $(u +f, v + g)$.

\section{Summary}

We have computed the correspondence between the SIAF polynomial
expansion coefficients, the SIP expansion coefficients, the
parity matrix $\mathtt{PC}$, and the pixel scale vector
$\mathtt{CDELT}$. To compute the SIP expansion from the SIAF
expansion, use Equations \ref{eqn:siaf_to_sip_x} and \ref{eqn:siaf_to_sip_y}
to compute $\sipA{i}{j}$ and $\sipB{i}{j}$ from $X_{ij}$, $Y_{ij}$,
$\mathtt{CDELT}$, and $\mathtt{PC}$. To incorporate further rotations
such that the sky and detector coordinates are rotated, apply a rotation
to the $\mathtt{CD}$ matrix and then compute the sky coordinates from
Equation \ref{eqn:sip_expansion} with the rotated $\mathtt{CD}$ matrix
(noting that $\mathtt{PC1\_1}$ and $\mathtt{PC2\_2}$ in Equations
\ref{eqn:siaf_to_sip_x}
 and \ref{eqn:siaf_to_sip_y} are defined in the unrotated frame when the
 sky and detector are aligned).

 \section{Inverse Transforms}

 Unfortunately, some libraries require the pre-computed inverse transforms.

 \subsection{Science to Detector}

The science to detector coordinate conversion is given in \S 4.1 of JTR1550 as
\begin{eqnarray}
XDet - XDetRef &=& DetSciParity\left(XSci - XSciRef\right)\cos\left(DetSciYAngle\right) \nonumber\\ 
&&- \left(YSci - YSciRef\right)\sin\left(DetSciYAngle\right)\nonumber
\end{eqnarray}
\begin{eqnarray}
YDet - YDetRef &=& DetSciParity\left(XSci - XSciRef\right)\sin\left(DetSciYAngle\right) \nonumber\\
&&+ \left(YSci - YSciRef\right)\cos\left(DetSciYAngle\right)\nonumber
\end{eqnarray}
\noindent
As before $u\equiv\left(XDet - XDetRef\right)$ and $v\equiv\left(YDet - YDetRef\right)$. Also, for
all NIRCam SCAs, $DetSciParity=-1$. Lastly, for NIRCam SCAs we have either $DetSciYAngle=0$ or
$DetSciYAngle=\pi$. We then can write
\begin{eqnarray}
u &=& DetSciParity\cos\left(DetSciYAngle\right)\left(XSci - XSciRef\right)
\end{eqnarray}
\begin{eqnarray}
v &=& \cos\left(DetSciYAngle\right)\left(YSci - YSciRef\right)
\end{eqnarray}
\noindent
Let us define $\NCu \equiv DetSciParity\cos\left(DetSciYAngle\right)$ and 
$\NCv \equiv \cos\left(DetSciYAngle\right)$. Note that $\NCu^{-1} = \NCu$ 
and $\NCv^{-1} = \NCv$, so we are left with
\begin{eqnarray}
u = \NCu^{-1} (XSci - XSciRef)
\end{eqnarray}
\noindent
and
\begin{eqnarray}
v = \NCv^{-1} (YSci - YSciRef).
\end{eqnarray}
\noindent
We can write this as
\begin{equation}
\left(\begin{array}{c}u\\v\end{array}\right) = \mathtt{PC}^{-1} \left[\begin{array}{c} (XSci - XSciRef)\\(YSci - YSciRef)\end{array}\right].
\end{equation}

 \subsection{Ideal to Science}
The inverse SIAF transform from Ideal to Science can be written
\begin{equation}
\label{eqn:siaf_xsci}
XSci - XSciRef = \sum_{i=1}^{degree} \sum_{j=0}^{i} Idl2SciCoefX_{i,j}XIdl^{i-j}YIdl^j
\end{equation}
\begin{equation}
\label{eqn:siaf_ysci}
YSci - YSciRef = \sum_{i=1}^{degree} \sum_{j=0}^{i} Idl2SciCoefY_{i,j}XIdl^{i-j}YIdl^j
\end{equation} 

\subsection{Inverse SIP Transform}
The inverse SIP transform takes the form
\begin{equation}
\left(\begin{array}{c}U\\V\end{array}\right) = \mathtt{CD}^{-1} \left(\begin{array}{c} x\\y\end{array}\right).
\end{equation}
Here, we have that 
\begin{eqnarray}
\label{eqn:sip_U}
u = U+F(U,V) = U+\sum_{p,q} \sipAP{p}{q} \,U^p \,V^q, & p+q \leq \mathtt{A\_ORDER},
\end{eqnarray}
\noindent
and
\begin{eqnarray}
\label{eqn:sip_V}
v = V+G(U,V) = V+\sum_{p,q} \sipBP{p}{q} \,U^p \,V^q, & p+q \leq \mathtt{B\_ORDER}.
\end{eqnarray}


\subsection{Matching Terms}
\begin{eqnarray}
u = \CDu^{-1} XIdl &+& \sum_{p,q} \sipAP{p}{q} \,\left(\CDu^{-1} XIdl\right)^p \,\left(\CDv^{-1} YIdl\right)^q \\
&=& \NCu^{-1} (XSci-XSciRef)\\
&=& \NCu^{-1} \sum_{i=1}^{degree} \sum_{j=0}^{i} Idl2SciCoefX_{i,j}XIdl^{i-j}YIdl^j
\end{eqnarray}
\noindent
Let's write out the third order expansion
\begin{eqnarray}
\CDu^{-1} x + \sipAP{1}{0} U + \sipAP{0}{1}V + \sipAP{2}{0}U^2 + \sipAP{0}{2}V^2 + \sipAP{1}{1}UV + \sipAP{3}{0}U^3 + \sipAP{0}{3} V^3 \\
+ \sipAP{2}{1}U^2V + \sipAP{1}{2}UV^2\\
=  \NCu^{-1} \siafXP{0}{0}  + \NCu^{-1} \siafXP{1}{0} x +  \NCu^{-1} \siafXP{1}{1} y + \NCu^{-1} \siafXP{2}{0} x^2 \\
+ \NCu^{-1} \siafXP{2}{1} xy + \NCu^{-1} \siafXP{2}{2} y^2 + \NCu^{-1} \siafXP{3}{0} x^3 \\
+ \NCu^{-1} \siafXP{3}{1} x^2 y + \NCu^{-1} \siafXP{3}{2} x y^2 + \NCu^{-1} \siafXP{3}{3} y^3   
\end{eqnarray}
\noindent
This is complicated by the fact that $\siafXP{0}{0}\ne0$, but we can handle that separately. It is also the case that $\siafXP{1}{0}\ne X_{10}^{-1}$. This inequality means that $\sipAP{1}{0}\ne0$.  It's also the case that in this definition, the units of all the $\siafXP{i}{j}$ are
different, and we will have to include powers of $\mathtt{CDELT}$.



Let's match the terms

% X10 is 0.031139270573203783 == CDELT1; 1/CDELT2 = 32.1137901303484
% Y11 is 0.03132231734459833 == CDELT2; 1/CDELT1 = 31.926118013501778
% XT00 is 0.0285276723582512, 
% XT10 is 32.113225595930636
% YT00 is -0.025760787593753776
% YT10 -0.026636640388253308
% YT11 is 31.925227769160344
%note for Forward, X00 is zero....

\begin{eqnarray}
\label{eqn:matched_inv}
\sipAP{0}{0} &\equiv& \NCu^{-1} \siafXP{0}{0}\\
\sipAP{1}{0} &\equiv& \mathtt{CDELT1}\siafXP{1}{0} - 1 \\
\sipAP{0}{1} \CDv^{-1} y &\equiv& \NCu^{-1} \siafXP{1}{1} y \\
\sipAP{2}{0} \CDu^{-2} x^2 &\equiv& \NCu^{-1}  \siafXP{2}{0} x^2\\
\sipAP{0}{2} \CDv^{-2} y^2 &\equiv& \NCu^{-1} \siafXP{2}{2} y^2\\
\sipAP{1}{1} \CDu^{-1} \CDv^{-1} xy &\equiv&  \NCu^{-1} \siafXP{2}{1} xy \\
\sipAP{2}{1} \CDu^{-2}\CDv^{-1} x^2y &\equiv& \NCu^{-1} \siafXP{3}{1} x^2 y\\
\sipAP{1}{2} \CDu^{-1}\CDv^{-2} xy^2 &\equiv& \NCu^{-1} \siafXP{3}{2} xy^2 \\
\sipAP{3}{0} \CDu^{-3} x^3&\equiv& \NCu^{-1}  \siafXP{3}{0} x^3 \\
\sipAP{0}{3} \CDv^{-3} y^3&\equiv& \NCu^{-1}  \siafXP{3}{3} y^3.
\end{eqnarray}
\noindent
This can be written succinctly as
\begin{equation}
\sipAP{i}{j} = \siafXP{(i+j)}{j} \NCu^{(i+1)} \NCv^{j} \mathtt{CDELT1}^i \mathtt{CDELT2}^j,
\end{equation}
\noindent
and correspondingly we have
\begin{equation}
\sipBP{i}{j} = \siafYP{(i+j)}{j} \NCu^{i} \NCv^{(j+1)} \mathtt{CDELT1}^i \mathtt{CDELT2}^j.
\end{equation}
\end{document}